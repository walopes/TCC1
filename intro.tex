\vspace{-1cm}
\chapter{Introdução} \label{cap:Introducao}

\vspace{-1.5cm} %%%%%%%% Para deixar a introdução no formato padrão
Este capítulo está dividido da seguinte forma: A Seção \ref{Consideracoes_iniciais} apresenta uma visão geral do tema abordado, 
a Seção \ref{Justi} trata da justificativa desta pesquisa, a Seção \ref{limitacoes} trata as limitações do 
projeto e a Seção \ref{objetivos} apresenta os objetivos a serem alcançados neste trabalho. 
% e a seção \ref{organizacao} traz uma visão geral sobre os capítulos posteriores.

%%%%%%%% Utilizar para dar os dois 'enter' antes de seção/ parágrafo
\begin{spacing}{2}\end{spacing} % Está funcionando
\section{Considerações iniciais} \label{Consideracoes_iniciais}    

A robótica é uma das áreas mais promissoras da engenharia, tendo aplicabilidade em várias áreas: de médicas a 
aeroespaciais, buscando oferecer produtividade e flexibilidade à sua aplicação. 
Atualmente é difícil encontrar atividades industriais que não possuam um sistema robótico ou 
automatizado, seja total ou parcial.\par


Devido à ampla aplicabilidade e utilidade que os robôs apresentam foram criadas competições, que visam estimular e 
contribuir com a pesquisa na robótica, tais como:
\begin{itemize}
 \item A \citeonline{robogames} que também é conhecida como ``Olimpíada dos robôs'', em 
que são disputadas mais de cinquenta categorias;
 \item A VEX \citeonline{vex}, a maior competição de robótica do mundo, que contou com 
 1075 times e mais de 15.000 participantes em sua última edição \cite{vex_guiness};
 \item A \citeonline{robocup}, que em 2016 foi sediada em Leipzig, Alemanha;
 \item A \citeonline{robocore}, com eventos realizados no Brasil.\par
\end{itemize}

O WinterChallenge, realizado anualmente pela Robocore, em São Paulo, 
é um dos maiores eventos de robótica da América Latina, contando com a participação de vários países e teve 
mais de mil competidores e cerca de quinhentos robôs na edição de 2016 \cite{maua}.\par

Uma das categorias disputadas é a dos seguidores de linha, na qual os robôs devem seguir, de maneira autônoma, 
um trajeto que é determinado por uma linha. Nessa categoria, se destacam os competidores do Japão, na competição Robogames, 
e do México, na Robocore, estes obtendo os três primeiros lugares na competição WinterChallenge na categoria 
Seguidor de Linha - Pro \cite{winter}. Competindo nesta mesma categoria, 
a equipe Patobots, da Universidade Tecnológica Federal do Paraná 
(\sigla{UTFPR}{Universidade Tecnológica Federal do Paraná}) - 
Câmpus Pato Branco, conquistou o 5º e 6º lugar, com os robôs \textit{Alpha project} e 
\textit{Robbie 3.0}, respectivamente.\par

Com base nesse contexto, este trabalho propõe a construção de um protótipo de um robô seguidor de linha
e o seu respectivo controle, 
visando participar de competições de robótica, como a Robocore.

%% Ficou um pouco forte, eu acho, mas foi solicitação da prof. Kathya para deixar desta forma.
O desenvolvimento deste trabalho contribuirá com a pesquisa que é feita na Universidade Tecnológica Federal do Paraná 
- Câmpus Pato Branco, na área de robótica móvel, em que já foram produzidos 
os trabalhos de \citeonline{allan} e \citeonline{alemao}, os quais servem de base para este trabalho.

\begin{spacing}{2}\end{spacing}
%%%%%%%%%%%%%%%%%%%%%%%%%%%%%%%%%%%%%%%%%%%%%%%%%%%%%%%%%%%%%%%%%%%%%%%%%%%%%%%%%%%%%%%%%%%%%%%%%%%%%%%%%%%%%%%%%%%%%%%%%%
\section{Justificativa} \label{Justi}

O trabalho de \citeonline{allan},
utilizou-se de controle híbrido, o qual 
integrou dinâmicas de controle discreto, como detecção de marcas laterais e a faixa central na pista, e contínuo, 
como o controlador Proporcional Integral Derivativo (\sigla{PID}{Proporcional Integral Derivativo}). Segundo o autor, 
o robô funcionou de acordo com o esperado para um percurso dentro das normas da Robocore, 
tendo alguns problemas relacionados à detecção das marcas laterais quando a pista estava com uma inclinação 
maior que 5º (graus).

O trabalho de \citeonline{alemao} desenvolveu um robô híbrido, realizando um estudo sobre os resultados obtidos por controladores
 PID e \textit{Fuzzy} (lógica difusa), sendo que o controlador PID apresentou melhor desempenho. Segundo o autor, não foi 
obtido sucesso considerável com a técnica \textit{Fuzzy}, que necessitava de um processador com poder computacional maior 
do qual foi utilizado, com memória suficiente para implementar as variáveis de controle.
 Devido às dificuldades encontradas, a comparação dos métodos foi 
realizada no robô 3pi, da Pololu, %\textsuperscript{TM}, %%% comando para o símbolo TM (Trademark)
em que o autor conseguiu uma velocidade de 1 m/s em retas. \par

Comparando os sistemas de controle desenvolvidos verificou-se que o Fuzzy
necessita de um processador bem mais eficiente comparado ao PID, com memória
suficiente para implementar as variáveis de controle, além de maior poder
computacional para processamento matemático, possibilitando o sistema de controle
compensar o erro em tempo hábil.

Com base nos trabalhos de \citeonline{allan} e \citeonline{alemao}, é proposta a modelagem de um novo 
\textit{hardware}, utilizando-se de técnicas de controle híbrido, o qual combina dinâmicas discretas (orientadas a 
eventos) e contínuas (orientadas a tempo) \cite{cassandras}. Também 
propõe-se a utilização de um microcontrolador com Unidade de Ponto-Flutuante 
(\sigla{FPU}{\textit{Floating-Point Unit}}), que pode facilitar a implementação de técnicas mais complexas e 
que exigem maior poder computacional e memória.


\begin{spacing}{2}\end{spacing}
%%%%%%%%%%%%%%%%%%%%%%%%%%%%%%%%%%%%%%%%%%%%%%%%%%%%%%%%%%%%%%%%%%%%%%%%%%%%%%%%%%%%%%%%%%%%%%%%%%%%%%%%%%%%%%%%%%%%%%%%%%
\section{Limitações do Trabalho}\label{limitacoes}

 
Este trabalho apresenta as seguintes limitações:
\begin{enumerate}
 \item Devido à complexidade que é o projeto e a implementação de um robô, este trabalho não se preocupará com 
 o desenvolvimento mecânico do dispositivo, sendo que a estrutura mecânica da mesma será confeccionada 
 sobre uma placa de circuito  impresso;
  \item Devido à dificuldade em encontrar e adquirir peças de alto desempenho, poderão ser utilizadas peças de menor custo, 
  as  quais podem reduzir as capacidades do robô;
 \item O projeto do robô seguidor de linha se aterá ao funcionamento em pistas que seguem as normas da Robocore, podendo 
 apresentar restrições de comportamento e até mesmo não funcionar, caso a pista não esteja no padrão estabelecido. 
\end{enumerate}


\begin{spacing}{2}\end{spacing}
%%%%%%%%%%%%%%%%%%%%%%%%%%%%%%%%%%%%%%%%%%%%%%%%%%%%%%%%%%%%%%%%%%%%%%%%%%%%%%%%%%%%%%%%%%%%%%%%%%%%%%%%%%%%%%%%%%%%%%%%%
\section{Objetivos}\label{objetivos}


\begin{spacing}{2}\end{spacing}
\subsection{Objetivo geral}

Projetar e implementar um protótipo de um robô seguidor de linha, com velocidade máxima próxima a 2 m/s, que seja autônomo,
através da utilização de controle 
híbrido, aperfeiçoando as técnicas desenvolvidas por \citeonline{alemao}.

\begin{spacing}{2}\end{spacing}
\subsection{Objetivos específicos}

\begin{itemize}

 \item Projetar e confeccionar a estrutura do protótipo, visando atender as dimensões especificadas pela Robocore;

  \item Projetar o condicionamento de sinais para os dispositivos a serem utilizados, permitindo 
  uma boa precisão na leitura dos sensores;
 
 \item Implementar o controlador PID, de modo a obter um controlador robusto e estável;
 
 \item Realizar testes com o protótipo em pistas que sigam as normas da Robocore;
 
 \item Implementar um sistema de telemetria, visando obter informações em tempo real do robô;
 
 \item Comparar os resultados obtidos com o de \citeonline{alemao}. 
 
 % A Kathya falou que não é necessário o item de baixo
 %\item Estudar a viabilidade de aplicação de inteligência artificial.
\end{itemize}


