\chapter{Recursos de Hardware e Software} \label{cap:recur}

Neste capítulo serão apresentados os principais recursos de \textit{hardware} e \textit{software} utilizados nesse projeto, bem como a origem destes recursos.


\section{Recursos de Hardware} \label{sec:recurhard}

Os recursos de \textit{hardware} necessários englobam o quadricóptero, o sistema de comunicação e a estação base.

O quadricóptero pode ser divido em duas partes: estrutura física e placa de controle. Os componentes da estrutura física são:

\begin{itemize}
\item 1x Chassi de 45cm diâmetro 
\item 4x Motor brushless 5000kV 
\item 4x Hélice 9x4,7 GWS 
\item 4x \sigla{ESC}{Eletronic Speed Control} 30A 
\item 1x Bateria 4000mAh 7,4V
\end{itemize}

estes componentes foram emprestados pelo prof. Hugo Vieira, orientador desse trabalho. Também foi emprestada uma placa de controle, chamada ``KK multicopter'', porém essa é uma placa de baixo desempenho e espera-se substituí-la por uma melhor. O desejável seria construir a própria placa de controle, porém devido ao encapsulamento \sigla{SMD}{Surface Mounted Device} utilizado nos sensores MEMS, a montagem dessas placas requer o uso de equipamentos específicos, inviáveis para esse projeto.

Até o momento não há disponibilidade de nenhum dos componentes do sistema de comunicação, todos deverão ser adquiridos. Eles são:

\begin{itemize}
\item 2x Módulo transceptor de RF
\item 1x USB \textit{dongle}
\item 1x Rádio controle 4 ou mais canais
\item 1x Receptor 4 ou mais canais
\end{itemize}

A estação base é um computador, desktop ou portátil, recente, com sistema operacional Windows ou Linux. Será usado um computador próprio.


\section{Recursos de Software}

Alguns recursos de software utilizados dependerão das alternativas de hardware escolhidas e só poderão ser definidas posteriormente. Inicialmente serão utilizados os seguintes:

\begin{itemize}
\item Matlab ou Octave: simulações do sistema de controle, coleta de dados do quadricóptero. Disponíveis na UTFPR e gratuito, respectivamente.
\item Eagle: criação de diagramas eletrônicos e placas de circuito impresso. Gratuito.
\item Astah community ou Dia: edição de diagramas UML e fluxogramas. Ambos gratuitos.
\end{itemize}






