
%\chapter{Recursos de Hardware e Software} \label{cap:materiais}
%\chapter{Materiais} \label{cap:materiais}
\chapter{Materiais} \label{cap:materiais}
\vspace{-2cm}
Neste capítulo serão apresentados os materiais a serem utilizados neste projeto.\newline
%\vspace{1cm}

%%%%%%%%%%%%%%%%%%%%%%%%%%%%%%%%%%%%%%%%%%%%%%%%%%%%%%%%%%%%%%%%%%%%%%%%%%%%%%%%%%%%%%%%%%%%%%%%%%%%%%%%%%%%%%%%%
\section{Microcontrolador} \label{cap:micro}

Será utilizado o \textit{Kit} de desenvolvimento NUCLEO-F303K8, da ST Microelectronics, %\textsuperscript{TM}, 
que possui as seguintes especificações básicas \cite{stm303}:

\begin{itemize}
 \item Microprocessador de arquitetura \textit{Advanced RISC Machine} (\sigla{ARM}{\textit{Advanced RISC Machine}}) Cortex-M4 de 32 bits com FPU;
 \item 72 MHz de frequência máxima de operação;
 \item Instruções de \textit{Digital Signal Processor} (\sigla{DSP}{\textit{Digital Signal Processor}});
 \item 90 DMIPS de desempenho;
 \item 64KB de memória \textit{Flash};
 \item 16KB de SRAM;
 \item 2 módulos \textit{Analog to Digital Converter} (\sigla{ADC}{\textit{Analog to Digital Converter}}) com até 21 canais;
 \item 11 módulos de temporizadores (\textit{timers}).
\end{itemize}%\newline
\vspace{0.5cm}

%%%%%%%%%%%%%%%%%%%%%%%%%%%%%%%%%%%%%%%%%%%%%%%%%%%%%%%%%%%%%%%%%%%%%%%%%%%%%%%%%%%%%%%%%%%%%%%%%%%%%%%%%%%%%%%
\section{Motores CC} \label{cap:motores}
Será utilizado o motor de corrente contínua (\sigla{CC}{Corrente contínua}) 
\textit{High-Power Carbon Brush} (\sigla{HPCB}{\textit{High-Power Carbon Brush}}) modelo 3041 da Pololu, %\textsuperscript{TM} 
\cite{pololu_motor}. Este motor possui alimentação de 12V, caixa de redução 10:1, 3000 
\sigla{RPM}{\textit{Revolutions Per Minute}} e eixo estendido, o qual permite o acoplamento do encoder magnético. 

\vspace{0.5cm}

%%%%%%%%%%%%%%%%%%%%%%%%%%%%%%%%%%%%%%%%%%%%%%%%%%%%%%%%%%%%%%%%%%%%%%%%%%%%%%%%%%%%%%%%%%%%%%%%%%%%%%%%%%%%%%%
\section{Ponte H} \label{cap:ponteh}
Será utilizada uma Ponte H para o controle de velocidade dos motores. O modelo que será utilizado é o TB6612FNG, da 
Toshiba, que é capaz de controlar até dois motores CC com corrente constante de 1,2A \cite{ponte}. 
A velocidade do motor é controlada por \textit{Pulse Width Modulation} (\sigla{PWM}{\textit{Pulse Width Modulation}}).


\vspace{0.5cm}

%%%%%%%%%%%%%%%%%%%%%%%%%%%%%%%%%%%%%%%%%%%%%%%%%%%%%%%%%%%%%%%%%%%%%%%%%%%%%%%%%%%%%%%%%%%%%%%%%%%%%%%%%%%%%%%
\section{Encoder magnético} \label{cap:encoder}
Um \textit{encoder} magnético é um transdutor de movimento, que converte movimentos em informações elétricas, %\cite{}.
sendo possível obter dados como posição e velocidade. Neste trabalho será utilizado o modelo 3081 da 
Pololu, %\textsuperscript{TM}, 
o qual realiza 12 contagens por revolução do eixo e é compatível com o motor 3041 
\cite{pololu_encoder}.

\vspace{0.5cm}

%%%%%%%%%%%%%%%%%%%%%%%%%%%%%%%%%%%%%%%%%%%%%%%%%%%%%%%%%%%%%%%%%%%%%%%%%%%%%%%%%%%%%%%%%%%%%%%%%%%%%%%%%%%%%%%
\section{Sensores de refletância} \label{cap:reflet}
O sensor de refletância é um dispositivo eletrônico que consiste de um \textit{Light Emitter Diode} 
\sigla{LED}{\textit{Light Emitter Diode}} e um fototransistor, medindo assim a refletância de uma superfície. Este circuito 
será utilizado para detectar a linha do percurso. 
O modelo que será utilizado nesse trabalho é o QRE1113P, da Fairchild %\textsuperscript{TM} Semiconductor\cite{reflet}.
Semiconductor\cite{reflet}.
\vspace{0.5cm}

%%%%%%%%%%%%%%%%%%%%%%%%%%%%%%%%%%%%%%%%%%%%%%%%%%%%%%%%%%%%%%%%%%%%%%%%%%%%%%%%%%%%%%%%%%%%%%%%%%%%%%%%%%%%%%%
\section{Placa de circuito impresso} \label{cap:chassi}
O \textit{chassi} do robô, ou seja, a estrutura deste, será confeccionada em uma placa de 
circuito impresso (\sigla{PCB}{\textit{Printed Circuit Board}}) de fenolite.
\vspace{0.5cm}


%%%%%%%%%%%%%%%%%%%%%%%%%%%%%%%%%%%%%%%%%%%%%%%%%%%%%%%%%%%%%%%%%%%%%%%%%%%%%%%%%%%%%%%%%%%%%%%%%%%%%%%%%%%%%%%
\section{Módulo bluetooth} \label{cap:bluetooth}
Será utilizado o módulo \textit{bluetooth} %\textregistered 
HC-05 para a telemetria. Este módulo possuiu a configuração 
mestre-escravo e comunicação \textit{Universal Asynchronous Receiver Transmitter} 
(\sigla{UART}{\textit{Universal Asynchronous Receiver Transmitter}}).

\vspace{0.5cm}


%%%%%%%%%%%%%%%%%%%%%%%%%%%%%%%%%%%%%%%%%%%%%%%%%%%%%%%%%%%%%%%%%%%%%%%%%%%%%%%%%%%%%%%%%%%%%%%%%%%%%%%%%%%%%%%
\section{Rodas} \label{cap:rodas}
Serão utilizadas rodas de poliuretano ou silicone.
\vspace{0.5cm}


%%%%%%%%%%%%%%%%%%%%%%%%%%%%%%%%%%%%%%%%%%%%%%%%%%%%%%%%%%%%%%%%%%%%%%%%%%%%%%%%%%%%%%%%%%%%%%%%%%%%%%%%%%%%%%%
\section{Bateria Lipo} \label{cap:bateria}
Será utilizada uma bateria do tipo Lítio-Polímero (\sigla{Li-Po}{Lítio-Polímero}) de duas céluas, 7,4V, 1300mAh e 
32,5A de corrente máxima de descarga, pois possui alta capacidade de corrente e densidade de carga.
\vspace{0.5cm}

%%%%%%%%%%%%%%%%%%%%%%%%%%%%%%%%%%%%%%%%%%%%%%%%%%%%%%%%%%%%%%%%%%%%%%%%%%%%%%%%%%%%%%%%%%%%%%%%%%%%%%%%%%%%%%%
\section{Conversor Step-Up} \label{cap:stepup}
O conversor \textit{Step-up} que será utilizado é o XL6009, que é um módulo elevador de tensão. Este circuito possui 
eficiência de 94\%, corrente e tensão de saída máxima de 3A e 35V, respectivamente \cite{stepup}.
\vspace{0.5cm}


%\newline


%%%%%%%%%%%%%%%%%%%%%%%%%%%%%%%%%%%%%%%%%%%%%%%%%%%%%%%%%%%%%%%%%%%%%%%%%%%%%%%%%%%%%%%%%%%%%%%%%%%%%%%%%%%%%%%
\section{Esfera deslizante} \label{cap:esfera}
Será utilizada uma esfera deslizante para sustentar a parte frontal do robô e manter os sensores de refletância em sua 
correta posição de funcionamento.
\vspace{0.5cm}


%\newline


%%%%%%%%%%%%%%%%%%%%%%%%%%%%%%%%%%%%%%%%%%%%%%%%%%%%%%%%%%%%%%%%%%%%%%%%%%%%%%%%%%%%%%%%%%%%%%%%%%%%%%%%%%%%%%%
%\section{Step-Down} \label{cap:stepdown}

%\newline




% https://www.pololu.com/product/3038

% https://www.pololu.com/product/3081



